\documentclass{article}

\usepackage{authblk}

\usepackage{fullpage}
\usepackage[portuguese]{babel}

\title{Personal Statement}
\author{Rodrigo Caetano de Oliveira Rocha}
%\affil{Pontif\'{i}cia Universidade Cat\'{o}lica de Minas Gerais (PUC Minas)}
\date{}

\begin{document}
\maketitle

Desde o in\'{i}cio do meu curso de gradua\c{c}\~{a}o, tenho tido como foco principal
seguir uma carreira acad\^{e}mica. Desde que eu descobri o prazer de
explorar as fronteiras da ci\^{e}ncia, tenho sido muito dedicado e apaixonado
com a pesquisa cient\'{i}fica. Independentemente do campo ou se a pesquisa possui um car\'{a}ter mais
te\'{o}rico ou pr\'{a}tico, eu realmente acredito que a ci\^{e}ncia \'{e} a \'{u}nica maneira
que n\'{o}s, seres humanos, podemos desenvolver ideias, teorias e tecnologias inovadoras capazes de mudar o mundo para melhor.

Durante meu bacharelado pela Pontif\'{i}cia Universidade Cat\'{o}lica de
Minas Gerais (PUC Minas), trabalhei como assistente de ensino para as disciplinas de C\'{a}lculo Diferencial
e Integral I e ​​tamb\'{e}m Algoritmos e Estruturas de Dados III. Essas
experi\^{e}ncias me ajudaram a desenvolver habilidades de ensino, que t\^{e}m sido muito importante
para mim como professor. Eu tamb\'{e}m pude aproveitar muitas oportunidades para trabalhar em projetos interessante
de desenvolvimento; por exemplo, alguns dos projetos que desenvolvi durante
o curso de gradua\c{c}\~{a}o incluem um compilador totalmente funcional para um subconjunto da
linguagem de programa\c{c}\~{a}o Oberon, um programa paralelo e distribu\'{i}do para a fatora\c{c}\~{a}o
de inteiros grandes, al\'{e}m de trabalhar em vis\~{a}o computacional e processamento de imagens durante uma
bolsa de inicia\c{c}\~{a}o cient\'{i}fica. Por essas e outras raz\~{o}es, conclu\'{i} meu
Bacharelado ({\em summa cum laude}) em Ci\^{e}ncia da Computa\c{c}\~{a}o, para o qual eu tive a
privil\'{e}gio de ser premiado com a Medalha de Honra ao M\'{e}rito pela PUC Minas. Pela gradua\c{c}\~{a}o
com honra ao m\'{e}rito, sendo o primeiro aluno da turma, eu tamb\'{e}m fui homenageado com o certificado
de Aluno Destaque pela Sociedade Brasileira de Computa\c{c}\~{a}o~(SBC).

Depois de receber meu diploma de Bacharel, eu me inscrevi para um curso de mestrado na
Universidade Federal de Minas Gerais (UFMG), que \'{e} considerado uma das
melhores universidades Brasileiras em Ci\^{e}ncia da Computa\c{c}\~{a}o. Conforme exigido pelo programa de mestrado,
trabalhei como assistente de ensino da disciplina de Redes de Computadores por um semestre.
Durante o mestrado, trabalhei com estruturas paralelas escal\'{a}veis ​​para aplica\c{c}\~{o}es intensivas de dados,
tamb\'{e}m conhecidas como aplica\c{c}\~{o}es {\em Big-Data}. Em particular, trabalhei no
desenvolvimento do Watershed, que \'{e} um sistema distribu\'{i}do para o
processamento de fluxos de dados em larga escala, inspirado no modelo de fluxo de dados. Eu projetei
uma abstra\c{c}\~{a}o modular para os canais de comunica\c{c}\~{a}o, permitindo
fluxos reutiliz\'{a}veis ​​e program\'{a}veis. Eu tamb\'{e}m integrei o Watershed com o Hadoop,
incluindo o YARN para escalonamento de tarefas e gest\~{a}o de recursos de cluster,
HDFS como um sistema de arquivos distribu\'{i}dos, e Zookeeper para a coordena\c{c}\~{a}o e
sincroniza\c{c}\~{a}o distribu\'{i}da.
Este trabalho resultou em um artigo publicado em um
workshop internacional~\cite{rocha2014watershed}. Esse artigo foi selecionado como
um dos melhores trabalhos do evento, onde fomos convidados a
submeter uma vers\~{a}o estendida para uma Edi\c{c}\~{a}o Especial da Revista {\em Concurrency and
Computation Practice and Experience} (ainda em fase de revis\~{a}o).

Eu tamb\'{e}m trabalhei em um projeto secund\'{a}rio com um professor do Departamento de
Matem\'{a}tica da UFMG, onde foi desenvolvido um algoritmo distribu\'{i}do para a detec\c{c}\~{a}o de ciclos
em grafos de grande escala. Fomos capazes de derivar uma prova formal da correc\c{c}\~{a}o do
algoritmo distribu\'{i}do, al\'{e}m de fornecer tamb\'{e}m uma an\'{a}lise de seu
desempenho te\'{o}rico. O algoritmo tamb\'{e}m foi implementado usando a plataforma do GraphChi,
que possui uma estrutura de execu\c{c}\~{a}o utilizando mem\'{o}ria secund\'{a}ria para o processamento
de grafos de grande escala. Este trabalho resultou em um artigo publicado~\cite{rocha2015cycles} com
men\c{c}\~{a}o honrosa, por estar entre as cinco melhores trabalhos
apresentados no simp\'{o}sio (para o qual ganhei um certificado).

Depois de terminar meu mestrado, estou trabalhando na PUC Minas como um
Professor Assistente, onde tenho lecionado para os cursos de Ci\^{e}ncia da Computa\c{c}\~{a}o e v\'{a}rios
Sistemas de Informa\c{c}\~{a}o, incluindo as disciplinas de Programa\c{c}\~{a}o Paralela e Teoria e
Algoritmos em Grafos. Eu tamb\'{e}m estou colaborando como co-orientador de
alunos de inicia\c{c}\~{a}o cient\'{i}fica, o que resultou em dois
artigos~\cite{saffran2015apriori,pereira2015stencilbench} que foram publicados em
um workshop de inicia\c{c}\~{a}o cient\'{i}fica, um dos quais~\cite{pereira2015stencilbench}
tamb\'{e}m premiado men\c{c}\~{a}o honrosa (ganhando um certificado por estar entre os melhores
trabalhos do workshop). Al\'{e}m desses artigos, tamb\'{e}m enviamos para a
revista {\em Concurrency and Computation Practice and Experience} um outro artigo sobre
uma t\'{e}cnica de {\em tiling} autom\'{a}tico, usando algoritmos gen\'{e}ticos, aplicado ao padr\~{a}o est\^{e}ncil,
permitindo o processamento de grandes entradas em GPUs (em fase de revis\~{a}o).
Al\'{e}m disso, eu estive em tr\^{e}s bancas examinadoras de trabalhos finais de gradua\c{c}\~{a}o,
sendo uma experi\^{e}ncia extremamente valiosa para mim.

Com o prop\'{o}sito de construir uma carreira acad\^{e}mica s\'{o}lida, tenho a inten\c{c}\~{a}o de prosseguir meus
estudos de p\'{o}s-gradua\c{c}\~{a}o na Universidade de Lancaster. Meu interesse na Escola de
Inform\'{a}tica e Comunica\c{c}\~{o}es da Universidade de Lancaster \'{e} devido \`{a} sua
excel\^{e}ncia em pesquisa cient\'{i}fica com prest\'{i}gio e reconhecido mundialmente. Especificamente, a oportunidade de Ph.D oferecida pelo Dr. Zheng Wang da Universidade de Lancaster \'{e} claramente
relacionados \`{a} pesquisa de p\'{o}s-gradua\c{c}\~{a}o que eu pretendo explorar. Um programa de Ph.D.
na \'{a}rea de paraleliza\c{c}\~{a}o autom\'{a}tica me proporcionaria uma excelente
oportunidade de me desenvolver como pesquisador.
Caso eu seja aceito, certamente me dedicarei a fim de tirar o maior proveito de tal oportunidade, sendo parte
de um grupo l\'{i}der internacional em sistemas computacionais.

\nocite{*}
\bibliographystyle{plain}
\bibliography{ref}
\end{document}
