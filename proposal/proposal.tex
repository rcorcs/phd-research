\documentclass[a4paper,12pt]{article}

\usepackage{authblk}
\usepackage[sort]{natbib}

\usepackage[utf8]{inputenc}

\usepackage{fullpage}

%\usepackage{amsmath,amssymb}
%\usepackage{mathrsfs}
%\usepackage{mathtools}
%\usepackage{epsfig,amsfonts,amsthm,amssymb,latexsym,amsmath}
%\usepackage[noend]{algpseudocode}
%\usepackage{algorithm}

%\usepackage{caption}
%\usepackage{subcaption}

%\usepackage{color}
%\usepackage{graphicx}

\usepackage{indentfirst}

\newcommand{\etal}{et~al.}
\newcommand{\protosection}[1]{\vspace{1ex}\noindent\textbf{#1}}

\linespread{0.96}
\title{\textbf{Automatic Parallelization for Heterogeneous Computing}}
\date{Prospective PhD supervisor - Dr. Zheng Wang}
\author[]{Rodrigo Caetano Rocha}

\begin{document}
\maketitle

\begin{abstract}

\end{abstract}

\section{Introduction}

During the last several years, modern computers are increasingly parallel and
heterogeneous~\cite{mohanty12,misailovic13}.  With parallel architectures being
widely spread, parallel programming must extend beyond its traditional realm of
scientific applications~\cite{li09}.  The ultimate goal of a programmer in a
modern computing environment is to write scalable applications that can take
complete advantage of all available resources, since parallelization is an
important step that can have serious impact on performance and
power consumption~\cite{cockx10}.  However, parallel programming is known to be difficult
and error prone, imposing several challenges to the programmer, such as race
conditions, deadlocks, or even
non-determinism~\cite{cockx10,mccool10,mccool12}.  Thus, the difficulty of
developing parallel software is a key obstacle for an average programmer to
exploit the considerable computational power that different multiprocessors
offer~\cite{misailovic13}.

Although manual parallelization by expert programmers is known to usually
results in the most efficient parallel implementation, it is a costly and
time-consuming approach, besides requiring specialized knowledge about
parallelism and the target architecture.  As hardware parallelism increases in
scale with each generation and programming costs increase, automated
parallelizing technology becomes extremely attractive with the potential to
greatly reduce the costs while still profiting from the parallel
architecture~\cite{tournavitis09,wang14a}.

Automatic parallelization is an optimization that produces parallel code that
is semantically equivalent to a sequential code received as input.  This is a
difficult problem since the parallel code must be correct and also perform
efficiently on the target machine~\cite{williams99}.  Despite intense research
interest in the area, it has failed to deliver outside niche
domains~\cite{tournavitis09,wang14a}.  Improving the performance of programs
for multiprocessors presents a challenge because even when thread-level
parallelism exists, it is difficult for a compiler to analyze and identify all
true dependencies between potential parallel threads~\cite{hammond98}.

The traditional approach had been to concentrate on restructuring
computationally intensive loops using dependence analysis to determine which
loops can be parallelized.  This analytical approach has only proved successful
with a small and restricted-set of programs~\cite{williams99}. Traditional
static parallelism detection techniques are not effective in finding
parallelism due to limitations on the level of information that can be
confidently acquired during static analysis~\cite{tournavitis09,wang14a}.
Another limitation in traditional static parallelism relates to the combination
of automatic parallelism discovery and portable mapping.  Given that the number
and type of processors of a parallel system is likely to change from one
generation to the next, finding the right mapping for an application may have
to be repeated many times throughout an application’s lifetime, hence, making
automatic approaches attractive~\cite{tournavitis09,wang14a}.

Furthermore, although this area have been widely studied, there is still much
to be explored regarding automatic parallelization that also optimize for
energy consumption, embedded systems~\cite{cordes10} or other emergent
computing architectures~\cite{baskaran10}, such as massively parallel
processors including graphic processing units (GPUs), many integrated cores
(MICs), and others.

Considering that most programs are executed indefinitely many more times than
they are compiled, compilers can afford performing expensive optimization
during compilation time.  Previous work suggest different approaches for
automatic parallelization based on machine learning
techniques~\cite{tournavitis09,wang14a} and also evolutionary
computing~\cite{walsh95,walsh96,williams96,williams99}.  However, extensive
research into automatic parallelization over the years has lead to the
development of a large body of heuristic and analytical techniques which may be
included into such modern approaches~\cite{williams99}. 

\section{Related work}

In this section we present an overview of several work related to automatic
parallelization of sequential code, where we discuss several parallelization
approaches for both multi-core and heterogeneous systems.

\subsection{Automatic parallelism based on static analysis}

Numerous early work make use of static analysis techniques in order to
parallelize affine loops~\cite{kuck81,padua93,blume94,lim97}.
Kuck~\etal~\cite{kuck81} discuss the use of data dependence graphs for several
code transformations and optimizations, including loop distribution, loop
fusion, loop blocking, loop interchanging, etc.  Lim and Lam~\cite{lim97}
propose an algorithm for finding the optimal affine transformations that
maximize parallelism while minimizing synchronization in programs with
arbitrary loop nesting and affine data accesses.

Misailovic~\etal~\cite{misailovic13} present {\em QuickStep}. Guided by
performance measurements, static analysis, loop and memory profiling
information, and statistical accuracy tests on the outputs of test executions,
QuickStep automatically searches this space to find a parallel program that
maximizes performance subject to a specified accuracy goal.

QuickStep prioritizes the parallelization of loops that consume more execution
time, where it explores the parallelization space for each loop by repeatedly
applying accuracy and performance enhancing transformations.  Once the final
parallelization is obtained, QuickStep produces an interactive parallelization
report that a developer can navigate to evaluate the acceptability of this
parallelization and obtain insight into how the application responds to
different parallelization strategies.

Automatic parallelization based on static analysis is challenging and limited
in the type of code that they can
parallelize~\cite{bruening98,kennedy01,chen03}.  In order to identify parallel
regions of code they must use complex interprocedural analyses to prove that
the code has no data dependencies for all possible inputs.  Typical code
targeted by these compilers consists of nested loops with affine array accesses
written in a language with limited aliasing.  Large systems written in modern
languages usually contain multiple modules and memory aliasing, which makes
them not amenable to automatic parallelization.  Furthermore, code whose memory
access patterns are indeterminable at compile time due to dependence on program
inputs can be impossible for these compilers to parallelize~\cite{bruening98}.

PluTo~\cite{bondhugula08}

\subsection{Thread-level speculative parallelism}

In contrast to automatic parallelization that aims at generating a parallel
code, thread-level speculation (TLS) approaches performs sequential code
parallelization during runtime in a speculative manner.  It consists of
selecting regions of code to execute in parallel, relying on the hardware to
detect dependence violations and re-execute the conflicting sections such that
sequential execution semantics is preserved~\cite{hammond98,chen03,wu08}.

Although TLS overcomes limitations intrinsic with conservative compile-time
automatic parallelizing tools by extracting parallel threads optimistically and
only ensuring absence of data dependence violations at runtime.  TLS usually
requires a specialized hardware for monitoring memory accesses made by the
parallel threads and for restarting any thread that attempt to violate true
dependencies~\cite{krishnan98,hammond98,wu08}.  The overheads associated with
maintaining speculative state is a significant barrier for the use of
TLS~\cite{yiapanis13}.

While TSL is usually performed mainly by hardware, there also exists
software-based speculative parallelism, such as Softspec, which is the
combination of a compiler and runtime system that can target program
binaries~\cite{bruening98}.

\subsection{Machine learning based approach}

Wang and O'Boyle~\cite{wang09} propose a machine learning based approach for
mapping fine-grain parallelism of OpenMP programs to multi-cores, namely, Intel
Xeon and the Cell processors.  They use an artificial neural network (ANN)
model to predict the scalability of a program and select the optimal thread
number for it.  Afterwards, a multi-class Support Vector Machine (SVM) model is
used to select a scheduling policy, such as cyclic, dynamic, guided or block
scheduling. 

The approach proposed by Tournavitis~\etal~\cite{tournavitis09} and
Wang~\etal~\cite{wang14a}, integrates profile-driven parallelism detection and
machine learning based mapping in a single framework for semi-automated
parallelism, where the user is expected to approve those loops where
parallelization is likely to be beneficial, but correctness cannot be proven
conclusively.  They use profiling data to extract actual control and data
dependencies and enhance the corresponding static analyses with dynamic
information. Subsequently, they apply a previously trained prediction mechanism
to each parallel loop candidate and decide if and how the parallel mapping
should be performed. Finally, they generate parallel code using standard OpenMP
annotations.

\subsection{Automatic parallelism based on evolutionary computing}

Walsh and Ryan propose {\em Paragen}~\cite{walsh95,walsh96}, a precursor for
the use of evolutionary computing in automatic parallelization.  Paragen makes
no attempt to analyse the original code, instead it attempts to directly
performs both loops and instructions parallelization. The correctness of the
resulting parallel code are determined by the fitness function based on the
outputs produced by the original sequential code and the one produced by the
parallel code.

Williams and Williams~\cite{williams96,williams99} also propose an evolutionary
approach for automatic parallelization.  This work suggests two genetic
representation.  The {\em gene-transformation} where each gene represents a
loop transformation (such as loop-skewing, loop-interchange, loop-fusion,
loop-splitting, loop-reversal, etc.) and the chromosome has variable-length and
is composed by an ordered sequence of transformations to be applied.  The {\em
gene-statement} where each gene represents a single instruction in the source
code. For the gene-statement representation, no crossover operation is
performed and the loop transformations are encoded as mutation operators with
predefined probabilities, where the length of the chromosome can change
depending on the mutations applied. The fitness function perform an analysis of
the resulting code of each chromosome generating an estimated execution-time.
They conclude that the gene-statement representation is better than the
gene-transformations representation.

Evolutionary based approaches have an optimistic parallelization strategy,
experimentally validating the best individuals. Although this approach is not
limited by dependence analysis techniques, the evolutionary approach produces a
probabilistic result, only with statistical guarantees.  In
Section~\ref{subsec:perfvsacc} we present several other work that discuss
probabilistic and approximate solutions for a variety of problems.

\subsection{Automatic parallelization for heterogeneous systems}

Most studies focus on automatic parallelization for CPU only.  However, there
have been some recent work~\cite{leung09,baskaran10,amini12,govindarajan13}
toward parallelization for heterogeneous architectures, usually GPUs.

Leung~\etal~\cite{leung09} extend the Java virtual machine to detect affine
loops, in plain Java bytecodes, which can be parallelized and executed more
quickly on the GPU rather than on the CPU.  They present a cost model that
balances the higher raw performance of the available GPU against the cost of
transferring input and output data between main memory and GPU memory.  There
are several obstacles that need to be overcome when parallelizing Java bytecode
language, such as unstructured control flow, the lack of multi-dimensional
arrays, the precise exception semantics, and the proliferation of indirect
references.

Baskaran~\etal~\cite{baskaran08,baskaran10} propose a framework for automatic
parallelization and optimization of affine loops on GPUs. A polyhedral compiler
model of data dependence is used to perform program transformation for
efficient data access from GPU global memory.  Their framework performs
source-to-source transformations, generating CUDA from C code, optimized for
efficient data access, including tiling strategies.  They use the
PluTo~\cite{bondhugula08} polyhedral parallel tiling infrastructure and the
polyhedral code generator called CLooG~\cite{bastoul04}, which transforms a
polyhedral representation of a program and affine scheduling constraints into
final loop code.

Parallelizing techniques has several limitations, one such limitation is
regarding indirect array accesses.  Govindarajan and
Anantpur~\cite{govindarajan13} present an algorithm for executing on GPU, loops
with cross iteration dependencies due to indirect memory accesses.  The
algorithm described is able to compute memory dependencies at runtime, where
dependence computation and actual computation are performed in pipeline.

%\subsection{Evolutionary computing applied to other compiler optimizations}
%
%Stephenson~\etal~\cite{stephenson03} make use of genetic programming to
%optimize the priority functions associated with two compiler heuristics,
%namely, predicated hyperblock formation and register allocation, achieving
%significant speedups over standard baselines.

%%TODO: shrink this paragraph
Schulte~\etal~\cite{schulte14a} propose a post-compilation optimization
approach that leverages ideas from evolutionary computation (population-based
stochastic search), mutational robustness (random mutations yield independent
implementations of the same specification), profile-guided optimization
(performance measured on workloads) and relaxed notions of program semantics
(customizing software to particular runtime goals and environments provided by
a software engineer).  The proposed approach is a steady-state evolutionary
algorithm, which maintains a population of candidate optimizations (assembly
programs), using randomized operators to generate variations, and selecting
those that improve an objective function (the power model) while retaining all
required functionality expressed in a test suite.  Their results show that the
evolutionary approach is effective on multiple architectures, finding
hardware-specific optimizations, with an average reduction of 20\% in energy
consumption.

\subsection{Performance versus accuracy trade-off}
\label{subsec:perfvsacc}

Several work discuss probabilistic and approximate architectures in order to
design hardwares with improved performance regarding power consumption,
execution time, physical space and architectural
simplicity~\cite{palem09,palem12,lingamneni12,kirsch12}.  In contrast with
traditional approaches, approximate architectures does not attempt to correct
the errors introduced by components which are susceptible to perturbations.
Palem~\etal~suggest their work has the potential to address challenges and
impediments to Moore's law arising from material properties and manufacturing
difficulties, indicating that their shifts from the current-day deterministic
design paradigm to statistical and probabilistic designs of the
future~\cite{palem09,palem12}.  Palem~\etal~demonstrate a XOR gate with a
$3\times$ reduction in energy consumption for a 2.3\% decrease in the
probability of correctness.  Chakrapani~\etal~\cite{chakrapani08} present a
mathematically rigorous foundation characterizing the trade-off between the
energy consumed and the quality of solutions.

Some recent work also propose numerous code transformations aiming at the
trade-off between performance and accuracy.  For certain types of applications,
accuracy-aware program transformations may generate error-tolerant code that
may perform better than the original code~\cite{zhu12}.  The main
transformations are: $(i)$ {\em substitution transformations} replace parts of
a program with code that computes approximations of the output computed by the
original parts but with less computational overhead; $(ii)$ {\em sampling
transformations} produces a transformed code that performs the same computation
as the original code but only on a subset of the elements obtained by some
sampling policy; $(iii)$ {\em loop perforation} is another type of program
transformation which transforms a loop into a new loop where only a subset of
the original loop iterations is performed.

Misailovic~\etal~\cite{misailovic11} provide a probabilistic approach based
that guarantees for the accuracy produced by loop perforation, while reducing
execution time and resource consumptions.  The transformations are analyzed
based on the perforation noise, defined by the difference between the result
that the perforated computation produces and the result that the original
computation produces. The perforation noise is then used to produce the
probabilistic guarantee of the transformation's accuracy.

Sidiroglou~\etal~\cite{douskos11} show that perforating appropriately selected
loops can produce significant performance gains (up to a factor of seven
reduction in overall execution time) in return for small (less than 10\%)
accuracy losses.

\section{Research directions}\label{sec:research-dir}

It is widely known that detecting parallelism with static compilers in large,
general programs is challenging~\cite{kennedy01,chen03}.  In order to overcome
the limitations of automatic parallelization based on static analysis for
general multiprocessors, we intend to explore optimistic and evolutionary
approaches aided by profiling and machine learning techniques.  We aim at
approaches where it is possible to parallelize aggressively rather than
conservatively~\cite{chen03,williams96,williams99}. In this section we suggest
some possible research directions and their respective challenges.

\subsection{An evolutionary approach}

Automatic parallelization using evolutionary computing has an optimistic
approach, where the code is parallelized more aggressively.  Each individual of
a generated population can then be scored based on profiling techniques.  While
this approach is not limited by dependence analysis mechanisms, several
challenges need to be addressed, including: $(i)$ define a genetic
representation for the source code, the parallel transformations and
optimizations; $(ii)$ define a fitness function that score each individual
considering the trade-off between performance and accuracy; $(iii)$ measure,
with statistical confidence, the accuracy of the parallelized source code;
$(iv)$ efficiently profile each individual.

Efficiently profiling each individual is important for reducing the latency
involved in the overall parallelization process using the evolutionary
approach.  Some of the strategies for efficiently profiling the program
corresponding to each individual include: simulating or actually executing the
program; executing a transformed version of the individual's program after
performing sampling or loop perforations~\cite{douskos11,misailovic11,zhu12};
estimating the performance without actually executing the
program~\cite{fahringer95b,fahringer00,fahringer11}.

Besides the traditional {\em Darwinian evolution model} for randomly creating
the initial population, mutations and crossovers, it is also possible to use
the {\em learnable evolution model}~\cite{michalski00,cervone00}, where the
individuals are {\em genetically engineered}. In the {\em learnable evolution
model}, individuals can be created guided by static analysis, profiling, or
machine learning.

\subsection{A machine learning based approach}

In addition to being used for mapping parallelism to multi-core
processors~\cite{wang09,tournavitis09,wang14a}, machine learning can also be
used for parallelizing a sequential code.  Similar to the evolutionary
approach, once we have optimistically identified all once the optimization and
parallelization search space has been identified in an optimistic manner, by
analyzing the sequential code, machine learning techniques, such as
ANN~\cite{ohlsson93,wang99}, can be used for automatically selecting the best
parallelization.

Several work propose solutions using neural networks for a variety optimization
problems in general~\cite{ohlsson93,wang99,smith99}, which can be applied
either using supervised or unsupervised
learning~\cite{bengio07}.

\subsection{Solving dependence violations during runtime}

By modeling the automatic parallelization problem as an optimization problem,
defining a parallelization search space in an optimistic manner, both the
evolutionary and the machine learning based approaches will produce
probabilistic solutions.

If a probabilistic solution is unacceptable, we can use speculative techniques
for detecting dependence violations during runtime for re-execution of
conflicting sections providing precise solutions.  This dependence monitoring
mechanism can be implemented entirely in software, similar to
Softspec~\cite{bruening98}, allowing this technique to be used in general
parallel hardware.  Although this mechanism guarantee correctness, it adds some
overhead to the overall execution, which needs to be properly balanced for
performance.

It is also possible to integrate techniques from dynamic parallelization, such
as those used by the \emph{hybrid analysis}~\cite{rus03}. \emph{Hybrid
analysis} combines static and runtime analysis of memory references into a
single framework.  In the context of heterogeneous systems, Govindarajan and
Anantpur~\cite{govindarajan13} describe an algorithm which is able to compute
memory dependencies at runtime, where dependence computation and actual
computation are pipelined on a GPU.

\subsection{Parallelization for heterogeneous systems and energy consumption}

Although automatic parallelization have been widely studied, there is still
much to be explored regarding energy consumption, embedded~\cite{cordes10} or
emergent computing
architectures~\cite{leung09,baskaran10,amini12,govindarajan13}, such as
massively parallel processors including graphic processing units (GPUs) or many
integrated cores (MICs).

Several new challenges are added when parallelizing for heterogeneous systems.
In addition to identifying parallelizable loops, it must also decide, for each
loop, if the loop parallelization will be profitable, and if so, where it
should be executed, i.e., whether to execute it on the CPU, the GPU, or any
other available parallel processor~\cite{tournavitis09,wang14a,leung09}.  In a
heterogeneous system, besides the raw performance of each processor,
synchronization, data communication or even energy consumption must also be
considered when evaluating how the parallel mapping should be performed.

\bibliographystyle{abbrv}
\bibliography{ref}
\end{document}
