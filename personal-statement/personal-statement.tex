\documentclass{article}

\usepackage{authblk}

\usepackage{fullpage}

\title{Personal Statement}
\author{Rodrigo Caetano de Oliveira Rocha}
%\affil{Pontif\'{i}cia Universidade Cat\'{o}lica de Minas Gerais (PUC Minas)}
\date{}

\begin{document}
\maketitle

Since the very beginning of my undergraduate course I have been focused on
following an academic career.  Ever since I first discovered the pleasure of
exploring the frontiers of science, I have been really dedicated and passionate
with scientific research.  Regardless of the field or whether it is a more
theoretical or practical research, I truly believe that science is the only way
we humans can develop groundbreaking ideas, theories and technologies that
change the world.

During my Bachelor's degree at Pontif\'{i}cia Universidade Cat\'{o}lica de
Minas Gerais (PUC Minas), I worked as a Teaching Assistant in Differential and
Integral Calculus I and also Algorithms and Data Structures III. Those
experiences helped me develop teaching skills, which have been very important
for me as a professor. I also had many opportunities to work on interesting
programming endeavours; for example, some of the projects I developed during
the undergraduate course include a fully functional compiler for a subset of
the Oberon programming language, a parallel and distributed factorization of
big integers, besides working in computer vision and image processing during an
undergraduate research scholarship. For those and other reasons, I concluded my
Bachelor's degree (summa cum laude) in Computer Science, for which I had the
privilege of being awarded with the Merit Medal of Honour from PUC Minas.  For
graduating with honours, and first in my class, I also received the certificate
of Outstanding Student from the Brazilian Society of Computer Science
(Sociedade Brasileira de Computa\c{c}\~{a}o - SBC).

After receiving my Bachelor's degree, I applied for a Master's degree at
Universidade Federal de Minas Gerais (UFMG), which is considered to be among
the best Brazilian universities in Computer Science. As required by my Master's
degree, I worked as a Teaching Assistant in Computer Networks. During my
Master's degree I worked with scalable parallel frameworks for data-intensive
applications, also known as Big-Data applications. In particular, I worked on
the development of the Watershed framework, which is a distributed system for
processing large-scale data streams, inspired on the dataflow model. I designed
a modular abstraction for the stream communication channels, allowing for
reusable and programmable streams. I also integrated Watershed with the Hadoop
framework, including YARN for job scheduling and cluster resource management,
HDFS as a distributed file system, and Zookeeper for distributed coordination
and synchronization. This work resulted in a paper published in an
international workshop~\cite{rocha2014watershed}. That paper was selected as
one of the best papers on the presented event, so we have been invited to
submit an extended version for a Special Issue of the journal Concurrency and
Computation Practice and Experience (still under revision).

I also worked in a secondary project with a professor from the Department of
Mathematics at UFMG, where we developed a distributed algorithm for cycle
detection in large-scale graphs. We were able to derive a correctness proof of
this distributed algorithm in addition to also providing an analysis of its
theoretical performance.  The algorithm was also implemented using the GraphChi
framework, which is a disk-based execution framework for large-scale graph
processing. This work resulted in a published paper~\cite{rocha2015cycles} that
was mentioned with honours, for being among the five best papers in the
presented symposium (for which I earned a certificate).

After finishing my Master's degree, I have been working at PUC Minas as an
Assistant Professor, where I have been teaching several Computer Science and
Information Systems coursers, including Parallel Programming and Theory and
Algorithms in Graphs.  I have also been collaborating as a co-adviser of
students working as undergraduate researches, which resulted in two
papers~\cite{saffran2015apriori,pereira2015stencilbench} that was published in
an undergraduate research workshop, one of which~\cite{pereira2015stencilbench}
was also mentioned with honours (earning a certificate for being among the best
papers in the workshop). Besides those papers, I have also submitted to the
journal Concurrency and Computation Practice and Experience another paper about
an automatic tiling technique, using genetic algorithms, applied to stencil
computations, allowing for processing large inputs on GPUs (under revision).
In addition, I have been in three examining boards of undergraduate final
projects, which was an extremely valuable experience for me.

%At the present, I am working with optimizing compilers, automatic
%parallelization, parallel programming in heterogeneous architectures, incluing
%CPU, GPU and MPPA-256 (Massive Parallel Processing Architecture), and graph
%theory.

With the purpose of building a solid academic career, I intend to pursue my
postgraduate study at the Lancaster University.  My interest in School of
Computing and Communications is due to its prestigious and worldwide recognized
excellence in research and teaching. For all these reasons, I also believe I
would be a good addition to your graduate program; and if accepted, I will
certainly dedicate myself to make the best of such an opportunity.

\nocite{*}
\bibliographystyle{plain}
\bibliography{ref}
\end{document}
